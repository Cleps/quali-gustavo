


\chapter{Fundamentação Teórica}

\section{Processamento Digital de Imagens}

\section{Deep Learning}

\section{Informação de profundidade}
Sensores de profundidade estão cada vez mais embarcados em equipamentos amplamente difundidos como dispositivos de realidade aumentada (Occulus, Kinnect) e até mesmo em smartphones \cite{du2020depthlab}, principalmente as câmeras ToF, pois são capazes de desempenhar de maneira satisfatória mesmo com baixa potência \cite{branscombe2018microsoft}. De acordo com \cite{xie2021ultradepth}, a adoção de sensores de profundidade em smartphones tende a aumentar nos próximos anos, com diversas aplicações como tradução de linguagem de sinais \cite{park2021enabling} e sistemas de navegação mobile para pessoas com deficiência visual \cite{see2022smartphone}.

%sinal 57, gestos 86, e aumentada 97

Ainda segundo \cite{castellano2023performance}, cada uma das técnicas de aquisição de imagens de profundidade possui lados negativos que podem impactar os dados. Por exemplo, as câmeras ToF podem sofrer com invalidação de pixels próximos a cantos ou bordas de objetos devido à interferências entre os raios IR em superfícies descontínuas ou reflexivas \cite{hansard2012time}. Outros tipos de câmeras RGB-D mais comuns como o Microsoft Kinect ou Intel RealSense podem produzir valores inválidos em superfícies muito brilhantes ou reflexivas como espelhos, superfícies metálicas ou muito escuras \cite{zollhofer2019commodity}. Em ambientes internos, tais imagens podem conter até 50\% de dados faltantes. \cite{zhang2022indepth} \cite{zhang2018deep}. Pontos cuja medição é desconhecida são representados por pixels totalmente pretos ou totalmente brancos \cite{dourado2020multi}.
\section{Modelos de estimação de profundidade}
