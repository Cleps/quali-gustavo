


\chapter{Fundamentação Teórica}

\section{Processamento Digital de Imagens}

Processamento digital de imagens segundo \citeonline{jain1989fundamentals}, consiste na manipulação de imagens em formato digital utilizando algoritmos para extrair informações, melhorar a qualidade ou transformar dados visuais. Esta área engloba diversas técnicas que permitem o tratamento de imagens capturadas por dispositivos digitais, como câmeras e scanners, visando a otimização de aspectos como contraste, nitidez e remoção de ruído \cite{gonzalez2010processamento}. Além de aprimorar a percepção visual, essas técnicas são essenciais para a análise automática de imagens, facilitando a identificação e classificação de objetos, a medição de propriedades geométricas e a extração de padrões \cite{russ2006image}.

No processamento digital de imagens, os processos geralmente são abordados em três níveis distintos, cada um com um papel específico na análise e interpretação das imagens. O nível baixo trata de manipulações mais primitivas, realizando operações fundamentais como filtragem, aprimoramento de contraste e remoção de ruído. O nível médio foca na segmentação e extração de características, onde a imagem é dividida em regiões de interesse e características relevantes são identificadas e descritas. Finalmente, o nível alto envolve a interpretação e reconhecimento dos dados processados, onde algoritmos são aplicados para classificar objetos, reconhecer padrões e realizar decisões baseadas em informações extraídas dos níveis anteriores. Cada um desses níveis do processamento de imagens contribui de maneira distinta para a análise visual, formando uma cadeia de processamento que vai desde a manipulação básica até a interpretação complexa \cite{gonzalez2010processamento}. 

Ainda segundo \citeonline{gonzalez2010processamento} o processamento digital de imagens envolve uma série de passos que vão desde a aquisição até interpretação das imagens. Como pode ser visto na figura X, essas etapas incluem:


\begin{enumerate}
    \item \textbf{Aquisição de Imagem}: Este estágio trata da obtenção de imagens, que podem ser capturadas por dispositivos digitais ou provenientes de arquivos digitais já existentes. Neste processo, podem ser realizados ajustes iniciais, como modificar o tamanho da imagem.
    
    \item \textbf{Ajuste e Melhoria de Imagens}: Nesta fase, a imagem é manipulada para adequá-la a uma aplicação específica. Em alguns casos, como em imagens médicas, a melhoria pode não ser a abordagem mais apropriada.

\item \textbf{Correção de Imagens}: Este passo busca aprimorar a qualidade visual das imagens através de métodos baseados em modelos matemáticos ou probabilísticos para compensar a degradação da imagem.

\item \textbf{Processamento de Imagens em Cores}: Com o aumento do uso de imagens digitais na web, o processamento de imagens coloridas tornou-se fundamental. Trabalhar com cores facilita a extração de características relevantes de uma imagem.

\item \textbf{Processamento em Multiresolução}: Este método envolve a representação de uma imagem em diferentes níveis de resolução, permitindo uma análise detalhada em várias escalas.

\item \textbf{Compressão de Imagem}: Nesta fase, são aplicadas técnicas para armazenar imagens de maneira mais eficiente ou reduzir a largura de banda necessária para sua transmissão.

\item \textbf{Processamento Morfológico}: O processamento morfológico foca na extração e análise dos componentes da imagem para descrever suas formas.

\item \textbf{Divisão de Imagem}: A segmentação é um dos aspectos mais desafiadores no processamento digital de imagens. Ela consiste em dividir a imagem em partes ou objetos distintos.

\item \textbf{Representação e Descrição de Características}: Este processo, também conhecido como seleção de atributos, visa extrair características que forneçam informações quantitativas ou qualitativas para distinguir entre diferentes tipos de objetos.

\item \textbf{Identificação de Objetos}: O reconhecimento de objetos é o processo de atribuir rótulos a elementos presentes em uma imagem, como classificar um objeto como um "carro".

\item \textbf{Conhecimento Contextual}: Envolve o entendimento do domínio do problema, incluindo informações detalhadas sobre áreas específicas na imagem onde se espera encontrar dados relevantes.

    
\end{enumerate}


O processamento digital de imagens é amplamente aplicado em áreas como a medicina, para a análise de imagens de diagnóstico, na indústria, para o controle de qualidade de produtos, na segurança, para reconhecimento facial e monitoramento e entre outros \cite{gonzalez2010processamento}. %https://archive.org/details/imageprocessingh03edruss


\section{Deep Learning}

\section{Informação de profundidade}
Sensores de profundidade estão cada vez mais embarcados em equipamentos amplamente difundidos como dispositivos de realidade aumentada (Occulus, Kinnect) e até mesmo em smartphones \cite{du2020depthlab}, principalmente as câmeras ToF, pois são capazes de desempenhar de maneira satisfatória mesmo com baixa potência \cite{branscombe2018microsoft}. De acordo com \cite{xie2021ultradepth}, a adoção de sensores de profundidade em smartphones tende a aumentar nos próximos anos, com diversas aplicações como tradução de linguagem de sinais \cite{park2021enabling} e sistemas de navegação mobile para pessoas com deficiência visual \cite{see2022smartphone}.

%sinal 57, gestos 86, e aumentada 97

Ainda segundo \cite{castellano2023performance}, cada uma das técnicas de aquisição de imagens de profundidade possui lados negativos que podem impactar os dados. Por exemplo, as câmeras ToF podem sofrer com invalidação de pixels próximos a cantos ou bordas de objetos devido à interferências entre os raios IR em superfícies descontínuas ou reflexivas \cite{hansard2012time}. Outros tipos de câmeras RGB-D mais comuns como o Microsoft Kinect ou Intel RealSense podem produzir valores inválidos em superfícies muito brilhantes ou reflexivas como espelhos, superfícies metálicas ou muito escuras \cite{zollhofer2019commodity}. Em ambientes internos, tais imagens podem conter até 50\% de dados faltantes. \cite{zhang2022indepth} \cite{zhang2018deep}. Pontos cuja medição é desconhecida são representados por pixels totalmente pretos ou totalmente brancos \cite{dourado2020multi}.
\section{Modelos de estimação de profundidade}
